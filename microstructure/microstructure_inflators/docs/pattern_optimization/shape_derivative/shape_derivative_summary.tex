\documentclass[twocolumn,10pt]{article}

\usepackage[latin1]{inputenc}
\usepackage{amsmath, amssymb, amsfonts, amsthm}
\usepackage{upgreek}
\usepackage{amsthm}
\usepackage{fullpage}
\usepackage{graphicx}
\usepackage{cancel}
\usepackage{subfigure}
\usepackage{mathrsfs}
\usepackage{outlines}
\usepackage[font={sf,it}, labelfont={sf,bf}, labelsep=space, belowskip=5pt]{caption}
\usepackage{hyperref}
% \usepackage{minted}
\usepackage{enumerate}
\usepackage{titling}

\usepackage{fancyhdr}
\usepackage[title]{appendix}

\DeclareMathOperator{\sgn}{sgn}

%% \pagestyle{fancy}
%% \headheight 24pt
%% \headsep    12pt
%% \lhead{Shape Derivatives for Functions of Homogenized Elasticity Tensors}
%% \rhead{\today}
%% \fancyfoot[C]{} % hide the default page number at the bottom
%% \lfoot{}
%% \rfoot{\thepage}
%% \renewcommand{\headrulewidth}{0.4pt}
%% \renewcommand\footrulewidth{0.4pt}
\providecommand{\abs}[1]{\lvert#1\rvert}
\providecommand{\norm}[1]{\lVert#1\rVert}
\providecommand{\dx}{\, \mathrm{d}x}
\providecommand{\dA}{\, \mathrm{d}A}
% \providecommand{\vint}[2]{\int_{#1} \! #2 \, \mathrm{d}x}
% \providecommand{\sint}[2]{\int_{\partial #1} \! #2 \, \mathrm{d}A}
\renewcommand{\div}{\nabla \cdot}
\providecommand{\shape}{\Omega(p)}
\providecommand{\mesh}{\mathcal{M}}
\providecommand{\boundary}{\partial \shape}
\providecommand{\vint}[1]{\int_{\shape} \! #1 \, \mathrm{d}x}
\providecommand{\sint}[1]{\int_{\boundary} \! #1 \, \mathrm{d}A}
\providecommand{\pder}[2]{\frac{\partial #1}{\partial #2}}
\providecommand{\tder}[2]{\frac{\mathrm{d} #1}{\mathrm{d} #2}}
\providecommand{\evalat}[2]{\left.#1\right|_{#2}}
\newcommand{\defeq}{\vcentcolon=}
\newtheorem{lemma}{Lemma}

\makeatletter
\usepackage{mathtools}
\newcases{mycases}{\quad}{%
  \hfil$\m@th\displaystyle{##}$}{$\m@th\displaystyle{##}$\hfil}{\lbrace}{.}
\makeatother

%% + Abtin
\usepackage{fullpage}
\usepackage[usenames,dvipsnames]{color}
\usepackage{paralist}
\usepackage{prettyref}
\newrefformat{sec}{Section~\ref{#1}}
\newrefformat{tbl}{Table~\ref{#1}}
\newrefformat{fig}{Fig.~\ref{#1}}
\newrefformat{chp}{Chapter~\ref{#1}}
\newrefformat{eqn}{Eq.~\eqref{#1}}
\newrefformat{set}{Eq.~Set~\eqref{#1}}
\newrefformat{alg}{Algorithm~\ref{#1}}
\newrefformat{apx}{Appendix~\ref{#1}}
\newcommand\pr[1]{\prettyref{#1}}

\newcommand\note[1]{\textcolor{magenta}{\bf [ABT: #1]}}
\renewcommand\vec[1]{\ensuremath{\mathbf #1}}
\def\x{\vec{x}}
\def\y{\vec{y}}
\def\u{\vec{u}}
\def\ue{\vec{u}^\e}
\def\strain{\varepsilon}
\def\d{\, \mathrm{d}}
\def\R{\, \mathbb{R}}
\DeclareMathOperator*{\argmin}{argmin}
\newcommand\para[1]{\paragraph{#1.~}}


\begin{document}
The minimization problem is generically defined as \note{place-holder;
  needs constraints etc.}
\begin{equation}
  \label{eqn:min}
  \argmin_\text{admissible $\omega$} J(\omega),
\end{equation}
where $J$ is some objective function on the micro structure. Letting
$S$ denote the compliance tensor, we choose
\begin{equation}
  \label{eqn:compObjective}
  J(\omega) = \frac{1}{2}\norm{S^H(\omega) - S^*}^2_F,
\end{equation}
for a microstructure with shape $\omega$. There are several other
possible choices for the objective functional $J$, such as deviation
of elasticity tensor or error in displacement. In our setting, we are
interested in particular values of Poisson ratios and shear moduli
and the compliance.\note{Justification for use of compliance}

The microstructure boundary $\partial \omega$ is parameterized by a
vector $\vec{p}$, consisting of, for instance, wire mesh node offsets
and thicknesses. With proper assumptions, the derivative of $\partial
\omega$ with respect to $\vec{p}$ is given by
\begin{equation}
  \vec{v}_{p_\alpha}(\y,\vec{p}) \defeq
  \pder{\y}{p_\alpha}\quad\text{for } \y\in \partial\omega,
\end{equation}
and defines perturbation velocity fields over the boundary. Using
$\vec{p}$ the minimization problem can be written as
\begin{equation}
  \label{eqn:minParam}
  \argmin_\text{admissible $\vec{p}$} J(\vec{p})\text{ where }
  J(\vec{p}) = \frac{1}{2}\norm{S^H(\vec{p}) - S^*}^2_F.
\end{equation}

The derivative of the objective function is then
\begin{equation}
  \label{eqn:paramJacobian}
  \pder{J}{p_\alpha} = [S^H - S^*]: \pder{S^H}{p_\alpha} = [S^H - S^*]: \d S^H[\vec{v}_{p_\alpha}],
\end{equation}
where $\d S^H[\vec{v}_{p_\alpha}]$ is the shape derivative of $S^H$
with perturbation $\vec{v}_{p_{\alpha}}$.

\para{Shape derivative of elasticity tensor} The shape derivative of
the homogenized elasticity tensor for microstructure with shape
$\omega$ and perturbation $\vec{v}$ is defined as the G\^ateaux
derivative \cite{zolesio2001shapes}
\begin{equation}
  \d C^H[\vec{v}] \defeq \lim_{t\downarrow 0} \frac{C^H(\omega(t,\vec{v}))-C^H(\omega)}{t},
\end{equation}
where $\omega(t,\vec{v}) \defeq \{\x+t\vec{v} : \x \in \omega\}$. The
homogenized elasticity tensor \note{give ref}, with a bit of
manipulation, can be rewritten in the energy form as
\begin{equation}
  \label{eqn:EhEnergy}
  C^H_{ijkl} = \frac{1}{|Y|} \int_\omega (\vec{e}^{ij} +
  \strain(\vec{w}^{ij})) : C^\text{base} : (\vec{e}^{kl} +
  \strain(\vec{w}^{kl})) \, \mathrm{d} \y.
\end{equation}
Using this form, taking its variation with respect to a
\emph{permissible} boundary perturbation, integrating by parts, using
the divergence theorem, and the fact that $\vec{w}^{ij}$ satisfies the
cell problem, the shape derivative of $C^H$ can be written as
\begin{align}
  \label{eqn:elastShapeDer}
  \d C^H_{ijkl}[\vec{v}] =
  \frac{1}{|Y|} \int_{\partial \omega} &[(\vec{e}^{ij} + \strain(\vec{w}^{ij})) : C^\text{base}\\
    &: (\vec{e}^{kl} + \strain(\vec{w}^{kl}))] (\vec{v} \cdot \hat{\vec{n}}) \dA(\y).\notag
\end{align}

\para{Shape derivative of compliance tensor} The compliance tensor is
the inverse of elasticity tensor, i.e. $S:C=I$. Using direct
differentiation
\begin{equation}
  \label{eqn:compShapeDer}
  \d S^H[\vec{v}] = -S^H : \d C^H[\vec{v}] : S^H.
\end{equation}
Combining the results from \pr{eqn:paramJacobian},
\pr{eqn:elastShapeDer}, and \pr{eqn:compShapeDer}, one can compute $\pder{J}{p_\alpha}$.

\para{Numerical computation} The integrand in \pr{eqn:EhEnergy} is
cubic over each boundary element ($\vec{e}^{ij} +
\strain(\vec{w}^{ij})$ and $\vec{v}\cdot\hat{\vec{n}}$ are linear) and
we use quadrature that evaluate the cubic function exactly.

\bibliographystyle{plain}
\bibliography{References}

\end{document}
